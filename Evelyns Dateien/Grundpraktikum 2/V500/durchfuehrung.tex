\section{Versuchsaufbau und Durchführung}
\label{sec:Durchführung}
Der gesamte Versuchsaufbau befindet sich in der Abbildung \ref{fig:optischeraufbau}. Das Licht von der Kondensorlinse wird durch einen Spalt geleitet, mit einer Linse gebündelt und dann an einem Prisma gebrochen. 
Die Abbildungslinse muss so justiert werden, dass ein scharfes Bild entsteht. Mit Hilfe des Geradsichtprimas wird das Licht in seine Spektrallinien aufgespaltet, sodass sie zu unterscheiden sind. 
Dazu gibt es auch noch eine Mattscheibe, die vor die Photokathode gestellt werden kann, um die schwerer sichtbaren Spektrallinien zu beobachten. 

Zuerst wird die Gegenfeldmethode angewandt, um die Energie des ausgelösten Elektronen zu bestimmen. Für diese Methode werden $5$ verschiedene Spektrallinien untersucht. Es werden jeweils $10$ Messwerte für die 
Spannung $U$ und den Strom $I$ notiert, wobei mit abnehmender Spannung der Strom zunimmt.
Zu der anderen zweiten Messung wird eine Wellenlänge von $\lambda = \SI{578}{\nano\meter}$ gewählt, um den Photostrom in Abhängigkeit der Spannung zwischen der Anode und Kathode zu messen. Dafür muss die 
Spannung zwischen $\SI{-20}{\volt}$ und $\SI{+20}{\volt}$ hochgeregelt werden. Zwischen den Bereich $\SI{1}{\volt}$ und $\SI{2}{\volt}$ werden die Messwerte in zweier Schritte notiert. Insgesamt werden aber mehr als 30 Messwerte notiert.
\begin{figure}[h!]
	\centering
	\includegraphics[width=0.9\linewidth]{OptischerAufbau.jpg}
	\caption{Genereller Versuchsaufbau zur Untersuchung des Photoeffektes, \cite[4]{anleitung500}.}
	\label{fig:optischeraufbau}
\end{figure}
