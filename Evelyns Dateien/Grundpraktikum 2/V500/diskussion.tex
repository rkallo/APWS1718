\section{Diskussion}

Im ersten Versuchsteil wurden folgende Grenzspannungen ermittelt:

\begin{equation*}
\begin{aligned}
U_{\text{g,\,557\,nm}} &= (0{,}57 \pm 0{,}02)\,\symup{V} \\
U_{\text{g,\,546\,nm}} &= (0{,}71 \pm 0{,}01)\,\symup{V} \\
U_{\text{g,\,491,6\,nm}} &= (0{,}96 \pm 0{,}15)\,\symup{V} \\
U_{\text{g,\,435,8\,nm}} &= (1{,}27 \pm 0{,}04)\,\symup{V} \\
U_{\text{g,\,407,8\,nm}} &= (1{,}47 \pm 0{,}03)\,\symup{V} \\
\end{aligned}
\end{equation*}
Anhand der Fehlerwerte lässt sich erkennen, dass bis auf die cyane Linie der Wellenlänge $\lambda = 491{,}6\,\symup{nm}$ alle Ergebnisse recht genau sind. Zu sehen ist dies auch in den Plots, bei denen die Messwerte nur in wenigen Fällen etwas stärker von der Ausgleichsgerade abweichen. Die ungenaue 
Bestimmung der Grenzspannung der cyanen Linie hängt damit zusammen, dass aus messtechnischen Gründen weniger Werte zur Verfügung standen als für die anderen Linien. 

Im zweiten Versuchsteil wurden die Grenzspannungen gegen die Frequenzen aufgetragen. Trotz der wenigen Messdaten ist kann die lineare Regression als genau angenommen werden, da sich die Werte nah an der Ausgleichsgerade befinden. Die Austrittsarbeit 
$A_{\text{K}}$ und das Verhältnis $\frac{h}{e_0}$ betragen

\begin{equation*}
\begin{aligned}
A_{\text{K}} &= (1{,}71 \pm 0{,}15)\, \symup{eV} \\
\frac{h}{e_0} &= (4{,}3 \pm 0{,}2 )\cdot 10^{-15}\, \symup{\frac{Js}{C}}.\\
\end{aligned}
\end{equation*}
Auch hier ist zu sehen, dass die Fehler klein sind. Ein Vergleichswert für $\frac{h}{e_0}$ ergibt sich aus den Literaturwerten für $h$ und $e_0$ \cite{naturkonstanten}:

\begin{equation*}
\begin{aligned}
h &= 6{,}626\cdot 10^{-34}\, \symup{Js} \\
e_0 &= 1{,}602 \cdot 10^{-19}\, \symup{C} \\
\implies \frac{h}{e_0}_{\text{Literatur}} &= 4{,}136 \cdot 10^{-15}\, \symup{\frac{Js}{C}}.
\end{aligned}
\end{equation*}
Der im Versuch ermittelte Wert von $\frac{h}{e_0}$ weicht somit um $3{,}9\%$ von dem Literaturwert ab, was eine kleine Abweichung ist. Somit ist diese Messung genau. 
Die Austrittsarbeit ist vom Material abhängig, liegt aber in einem Bereich von etwa 1 bis 6\,eV \cite{austrittsarbeit}. Die als $A_{\text{K}}= (1{,}71 \pm 0{,}15)\, \symup{eV}$ bestimmte Austrittsarbeit liegt ebenfalls in diesem Bereich und ist damit
mit den Literaturwerten vergleichbar.


Der letzte Versuchsteil zeigt die schlechtesten Ergebnisse vor. Dass der Photostrom einen Sättigungswert erreicht ist nur ansatzweise zu erkennen, weshalb die verschiedenen Eigenschaften des Photoeffekts, die gezeigt werden sollen, nicht sehr deutlich 
zum Vorschein kommen. Zudem ist der Messbereich kleiner als vorgegeben, da messtechnisch keine weiteren Messungen ab $0{,}4\,\symup{V}$ möglich waren. Es ist anzunehmen, dass mit genaueren Versuchs- und Messgeräten bessere Ergebnisse erzielt werden können,
die die Vorgänge beim Photoeffekt anschaulicher machen.
