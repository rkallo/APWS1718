\section{Diskussion}

Der erste Versuchsteil ergibt mittlere Reichweiten der $\alpha$-Strahlung von

\begin{equation*}
\begin{aligned}
x_{1{,}5} &= (0{,}015 \pm 2{,}707\cdot 10^{-7})\,\symup{m}\\
x_{2}  &= (0{,}015 \pm 3{,}789\cdot 10^{-7})\,\symup{m}\\
\end{aligned}
\end{equation*}
und die zugehörigen Energien

\begin{equation*}
\begin{aligned}
E_{\text{1{,}5}} &= 2{,}86\,\symup{MeV} \\
E_{\text{2}} &= 2{,}86\,\symup{MeV}. \\
\end{aligned}
\end{equation*}
Es fällt auf, dass sich die berechneten Reichweiten nur etwa auf einen geringen Fehlerwert unterscheiden, sodass auch die entsprechenden Energien für beide Abstände gleich sind.
Dies wurde zu Beginn des Versuchs angenommen und kann hiermit also bestätigt werden.

Die Energieverluste

\begin{equation*}
\begin{aligned}
-\frac{\symup{d}E}{\symup{d}x_{1{,}5}} &= (76{,}48 \pm 3{,}04)\,\symup{\frac{MeV}{m}} \\
-\frac{\symup{d}E}{\symup{d}x_2} &= (62{,}22 \pm 4{,}04)\,\symup{\frac{MeV}{m}} \\
\end{aligned}
\end{equation*}
befinden sich etwa in der gleichen Größenordnung, weichen aber dennoch mit etwa $23\%$ voneinander ab. Da dieser Versuchsteil nicht genau genug durchgeführt worden ist, ist anzunehmen, dass hier genauere 
Ergebnisse zu erreichen sind.

Im letzten Teil des Versuchs soll das Histogramm der Messwerte mit einer Gauß- und einer Poissonverteilung verglichen werden. Vorab wurde angenommen, dass sich das Histogramm wie eine Poissonverteilung verhält,
da die Häufigkeit, mit der eine Zahl von Zerfällen stattfindet, eben dieser Verteilung entspricht.
In Abbildung \ref{fig:histogramm} lässt sich allerdings erkennen, dass die Messwerte eine etwas höhere Übereinstimmung mit der Gaußverteilung zeigen, was die Annahme damit nicht bestätigt. Dass mit dem Histogramm allerdings
keine klaren Aussagen getroffen werden können, könnte messtechnische Gründe haben, weshalb nicht ausgeschlossen werden sollte, dass eine genauere Messung in diesem Fall auch zu einem besseren Ergebnis führen könnte.