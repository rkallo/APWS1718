\section{Zielsetzung}
\label{sec:Zielsetzung}

Ziel des Versuches ist es die Reichweite von $\alpha$-Strahlung in Luft zu bestimmen, um daraus die Energie der Strahlung zu bestimmen.  Am Ende soll die Statistik des radioaktiven Zerfalls überprüft werden.

\section{Theorie}
\label{sec:Theorie}

Das Entstehen von $\alpha$-Teilchen basiert auf dem quantenmechanischen Tunneleffekt. Wird von der klassischen Physik ausgegangen dürfte es nicht möglich sein, dass ein Kern zerfällt, da er durch starke Wechselwirkung
zusammengehalten wird. Beim Tunneleffekt gibt es jedoch eine Zerfallswahrscheinlichkeit, bei der sich das Teilchen vom Kern löst. Veranschaulichen lässt sich dies  durch eine endlich hohe Potentialbarriere, die mit einer
von Null verschiedenen Wahrscheinlichkeit vom Teilchen überwunden werden kann.

Durchlaufen nun $\alpha$-Teilchen Luft, so kommt es zu Wechselwirkung mit den Luftatomen. Es kommt zu Energieverlust beziehungsweise Abgabe von Energie aufgrund von Ionisierungsprozessen, Anregung und Dissoziation von Molekülen. 

Der Energieverlust pro Wegstück ist somit von der Energie der $\alpha$- Strahlung und der Dichte der durchlaufenden Materie abhängig. Bei kleinen Geschwindigkeiten nimmt die Wechselwirkungswahrscheinlichkeit zu.
Da es unterschiedliche Wechselwirkungswahrscheinlichkeiten für alle Energiebereiche existieren, gibt es keine universelle Formel. Für hinreichend große Energiebereiche gibt es die Bethe-Bloch-Gleichung:
\begin{equation*}
-\frac{dE_\alpha}{dx} = \frac{z^2e^4}{4 \pi \epsilon_0 m_e}\frac{nZ}{v^2}\text{ln}\left(\frac{2 m_e v^2}{I}\right)
\end{equation*}
wobei $v$ die Geschwindigkeit der $\alpha$-Strahlung, $z$ die Ladung, $n$ die Teilchendichte, $I$ die Ionisierungsenergie des durchlaufenen Gases, $Z$ die Ordnungszahl sind. Für kleine Energien gilt diese Formel nicht mehr aufgrund von Ladungsaustauschprozesse. Deshalb wird empirisch eine gewonnene Kurve für eine $\alpha$-Strahlung in Luft verwendet. Es sollen nur Energien unter $\SI{2,5}{\mega\electronvolt}$ berücksichtigt werden, damit die folgende Beziehung gelten kann:
\begin{equation}
R_\text{m} = 3,1 \cdot E^{\frac{3}{2}}_{\alpha}
\label{eq:reichweiteenergie}
\end{equation}
wobei $R_\text{m}$ die mittlere Reichweite von $\alpha$-Teilchen ist. Diese beschreibt die Reichweite, die die Hälfte der Teilchen noch erreicht. Die Reichweite $R$ eines $\alpha$-Teilchens lässt sich durch:
\begin{equation*}
\int_0^{E_\alpha} \frac{dE_\alpha}{-dE_\alpha/dx}
\end{equation*}
bestimmen. Zu beachten ist noch, dass die Bethe-Bloch-Gleichung bei sehr kleinen Energien aufgrund der dann auftretenden Ladungsaustauschprozesse nicht mehr anwendbar ist.
Wenn die Temperatur und das Volumen konstant bleibt, so ist die Reichweite eines $\alpha$-Teilchens proportional zum Druck $p$. Durch diese Bedingungen wird eine Absorptionsmessung gemacht. Für die effektive Länge ergibt sich:
\begin{equation}
x = x_0 \frac{p}{p_0}
\label{eq:reichweite}
\end{equation}
wobei $p_0 = 1013\,\symup{mbar}$ der Normaldruck und $x_0$ der feste Abstand zwischen der $\alpha$-Strahler und Detektor ist. 