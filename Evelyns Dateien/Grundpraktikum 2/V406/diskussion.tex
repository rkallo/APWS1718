\section{Diskussion}

Beim Vergleich der Messwerte des ersten Einfach-Spaltes mit der Regression fällt auf, dass die Ausgleichsrechnung in etwa dem Verlauf der aufgenommenen Werte entspricht, aber dennoch Abweichungen zeigt.
Hier lässt sich vermuten, dass die Regression noch genauer vorgenommen werden kann, um bessere Ergebnisse zu erzielen.
Für den ersten Einfach-Spalt ergeben sich folgende Werte für die experimentelle und tatsächliche Spaltbreite:

\begin{equation*}
\begin{aligned}
\text{Experimentell:}\,\, b_{\text{es,1}} &= (0{,}0784\pm 0{,}0008)\,\symup{mm} \\
\text{Tatsächlich:}\,\, b_{\text{t1}} &= 0{,}075\,\symup{mm}.\\
\end{aligned}
\end{equation*}
Die Abweichung zwischen den beiden Werten beträgt $4{,}53\%$. Da dies eine recht kleine  Abweichung ist, kann diese Untersuchung als genau angenommen werden.

Die Regression der zweiten Messreihe zeigt beim Betrachten der Abbildung \ref{fig:es2} bessere Ergebnisse vor. Hier stimmt die Regression mehr mit den Messwerten überein als im vorherigen Versuchsteil.
Die Spaltbreite wird wie folgt bestimmt:
\begin{equation*}
\begin{aligned}
\text{Experimentell:}\,\, b_{\text{es,2}} &= (0{,}15279\pm 0{,}00116)\,\symup{mm} \\
\text{Tatsächlich:}\,\, b_{\text{t2}} &= 0{,}15\,\symup{mm}.\\
\end{aligned}
\end{equation*}
Hier lässt sich eine geringe Abweichung von $1{,}86\%$ feststellen. Somit ist auch diese Untersuchung genau.

Im letzten Versuchsteil ist zu erkennen, dass die Regression recht gut den Verlauf der aufgenommenen Messdaten darstellt. Die Beugungsfigur des Einfach-Spaltes hüllt die des Doppelspaltes im mittleren Teil bis auf kleine Abweichungen zunächst
ein, was für größer werdenden Schrittweiten jedoch nicht mehr der Fall ist. Es lässt sich die Vermutung anstellen, dass mit einer genaueren Versuchsdurchführung diese Eigenschaft des Einfach-Spaltes deutlicher werden würde.

Insgesamt hat der Versuch jedoch gute Ergebnisse geliefert, wie sich an den geringen Abweichungen der experimentellen Werte von denen des Herstellers erkennen lässt.
