\section{Zielsetzung}
\label{sec:Zielsetzung}
Das Ziel des Versuches ist die Bestimmung der Viskosität newtonscher Flüssigkeiten sowie die Temperaturabhängigkeit der Viskosität von destilliertem Wasser. 
Weiterhin soll mit Hilfe der Reynolds-Zahl untersucht werden, ob sich die Strömung laminar verhält. Außerdem soll auch die Apparaturkonstante der großen Kugel bestimmt werden.
\section{Theorie}
\label{sec:Theorie}
Wird eine Kugel betrachtet, welche in eine zähe Flüssigkeit fällt, so wird diese aufgrund der Schwerkraft $\vec{F_\text{G}}$ nach unten beschleunigt. Je schneller die Kugel sinkt, desto kleiner wird 
die Beschleunigung. Dem entgegen wirken noch die Reibungskraft $\vec{F_\text{R}}$ und die Auftriebskraft $\vec{F_\text{A}}$. Die Schwerkraft $\vec{F_\text{G}}$ wird dann von der Reibungskraft $\vec{F_\text{R}}$ 
kompensiert bis sich ein Kräftegleichgewicht gestellt hat und sich die Kugel weiterhin mit einer konstanten Geschwindigkeit $v$ bewegt.
Die Reibungskraft ist gegeben durch:
\begin{equation*}
F_\text{R} = 6\pi\eta v r
\end{equation*}
wobei $r$ der Radius der Kugel, $v$ die Geschwingkeit des Körpers und $\eta$ die Viskosität sind. 
Diese Gleichung wird als Stokessches Gesetz benannt.
In diesem Versuch wird das Kugelfall-Viskosimeter nach Höppler angewendet. In einem geringfügigen großen Rohrdurchmesser, gefüllt mit destilliertem Wasser, fällt eine Kugel mit einem kleinen Durchmesser. 
Es können sich durchaus Wirbel ausbilden, wenn die Kugel durch das senkrecht stehende Rohr fallen würde. In diesem Fall wird das Rohr leicht um einige Grade geneigt, um Wirbel zu vermeiden und damit die Kugel 
an der Rohrwand hinabgleiten kann. Die Viskosität $\eta$ wird durch die Fallzeit $t$, die Kugeldichte $\rho_{K}$ und des Wassers $\rho_{Fl}$, sowie durch die Apparaturkonstante $K$, die von der Fallhöhe als auch von der Geometrie 
des Kugel abhängt, bestimmt:
\begin{equation}
\label{eqn:Etat}
\eta=K\left(\rho_{K}-\rho_{Fl}\right)\cdot t
\end{equation}
Dabei hängt die Viskosität $\eta$ sehr stark von der Temperatur ab und lässt sich bestimmen durch die sogenannte Andradedschen Gleichung:
\begin{equation*}
\eta_{\text{T}}=A\cdot \symup{exp}\biggl(\frac{B}{T}\biggr)
\end{equation*}
wobei $A$ und $B$ die Konstanten der Gleichung sind. 
Es wird von einer turbulenten Strömung gesprochen, wenn sich in der Flüssigkeit Wirbel ausbilden und eine hohe Strömungsgeschwingkeit vorliegt. Bei einer laminaren Strömung bilden sich keine Wirbel 
aus und die Flüssigkeit läuft parallel zur Rohrachse. Um zu überprüfen, ob sich die Kugel in einer laminaren oder turbulenten Strömung befindet, wird die Reynoldszahl gebraucht. Diese ist gegeben durch
\begin{equation}
\label{eqn:Reynolds}
Re = \frac{\rho v d}{\eta}
\end{equation}
mit $\rho$ als Dichte der Flüssigkeit, $v$ als mittlere Geschwindigkeit der Kugel, $\eta$ die Viskosität der Flüssigkeit und $d$ der Durchmesser der Kugel sind. 
