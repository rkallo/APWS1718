\newpage 
\section{Diskussion}
Ziel dieses Versuches ist es, das magnetische Moment eines Permanentmagneten auf drei verschiedene Weisen zu bestimmen.
In jeder der Messmethoden gab es Fehlerquellen, die berücksichtigt werden müssen.
Zunächst die Gravitationsmethode. Hier können Fehler beim Vermessen der Billiardkugel mitsamt des Stiels auftreten. 
Zudem ist es möglich, dass die Abstände $r$ von der Masse bis zur Kugel nur ungenau bestimmmt werden. Diese ungenauen Abstände würden sich damit direkt auf die
Berechnung von $\mu_{\text{Dipol}}$ auswirken. 

Bei der Methode über die Schwingungsdauer könnte es sein, dass die Auslenkungen nicht bei jeder Messung gleich groß sind und die 
Durchführungen somit nicht einheitlich, da Periodendauern dadurch vergrößert oder verkleinert werden. Bei kleinen Periodendauern könnten
sich außerdem Fehler beim Stoppen der Zeit ergeben. 

Zuletzt gibt es auch bei der Präzessionsmethode einige Fehlerquellen. Zunächst muss die Kugel in Rotation und in einen stabilen Zustand
versetzt werden. Wenn dies gelingt, präzidiert die Kugel häufig schon, bevor der Strom richtig eingestellt wird. Außerdem kommt
es auch hier, wie in der Schwingungsmethode, vor, dass die Kugel nicht immer gleich stark ausgelenkt wird. Weiterhin ergeben sich 
auch Fehler beim Messen der Periodendauer, da es nicht immer genau ersichtlich ist, wann eine Periode erreicht wird. 

Ferner spielt bei allen Messmethoden eine Rolle wie lange der Strom eingeschaltet ist, weil beim Erwärmen des Spulendrahtes dessen
Widerstand ansteigt. Daraus würde folgen, dass das Magnetfeld schwächer wird, was sich auf die gesamte Berechnung der magnetischen
Momente der verschiedenen Messmethoden auswirkt. 

Die Ergebnisse dieses Versuchs sind die folgenden magnetischen Momente:

\begin{equation*}
\begin{aligned}
\mu_{\text{Dipol, Grav}} &= (0{,}34 \pm 0{,}04)\, \symup{A\cdot m^2} \\
\mu_{\text{Dipol, Schwing}} &= (0{,}368 \pm 0{,}017)\, \symup{A\cdot m^2} \\
\mu_{\text{Dipol, Präz}} &= (0{,}31 \pm 0{,}03)\, \symup{A\cdot m^2} 
\end{aligned}
\end{equation*}

Diese Werte weichen mindestens mit $7{,}61\%$ und maximal mit $18{,}71\%$ voneinander ab. Es ist zu erkennen, dass das magnetische Moment,
das über die Gravitationsmethode bestimmt wurde, den größten Fehlerwert aufweist. Dagegen hat die Schwingungsmethode den kleinsten und ist somit die genaueste der drei Messmethoden.
