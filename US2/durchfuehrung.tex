\section{Aufbau und Durchführung}

Zu dem Versuchsaufbau gehören ein Ultraschallechoskop, eine Sonde für das Impuls-Echo-Verfahren mit einer Frequenz von 2\,MHz und ein Rechner zur Datenverarbeitung und -darstellung.

Nachdem der zu untersuchende Acrylblock ausgemessen wird, werden die Störstellen in dem Block mit einem A-Scan bestimmt. Hierzu wird die Sonde über die betreffenden Stellen gehalten, wobei Wasser als Koppelmittel verwendet wird. Auf dem Rechner 
ist während der Untersuchung ein Graph zu sehen, aus dem die Abstände zu den Störstellen abgelesen werden können. Die Untersuchung wird für die gegenüberliegende Seite des Blocks wiederholt.
Zudem wird das Auflösungsvermögen analysiert, indem zwei kleine, benachbarte Fehlstellen ausgemessen werden.

Im nächsten Versuchsteil wird die Untersuchung der Bohrungen mit dem B-Scan durchgeführt. Dazu wird das Computerprogramm auf den B-Scan eingestellt und dann gestartet. Daraufhin wird die Sonde gleichmäßig über den Block geführt, woraus das Programm 
ein Bild aus Helligkeitsabstufungen generiert, aus welchem die Lage der Störstellen abgemessen werden können.

Zuletzt wird ein Herzmodell mit dem TM-Scan untersucht. Hierfür wird ein Doppelgefäß mit einer Membran, die mit einer Pumpe gewölbt werden kann, verwendet. Der obere Teil des Gefäßes wird etwa zu einem Drittel mit Wasser gefüllt und die Sonde so 
angebracht, dass sie gerade die Oberfläche berührt. Dann kann mit einem A-Scan die Laufzeit des Echos ermittelt werden. Mit dem TM-Scan erfolgt die Bestimmung des Herzvolumens und der Frequenz. Dazu wird mit der Pumpe gleichmäßig die Membran bewegt,
wodurch die Wasseroberfläche steigt und sinkt. Dies wird über den Computer im M-Mode aufgenommen und zu einem Bild verarbeitet, aus dem die Frequenz und das Herzvolumen bestimmt werden kann.