\section{Diskussion}
\label{sec:Diskussion}
Es lässt sich erschließen, dass die ermittelten Werte bei der Zeitkonstante nah beieinander liegen.
Es können jedoch systematische Fehler auftreten, da der Innenwiderstand des Generators bei allen Versuchen vernachlässigt wird. Der genauen Wert der Zeitkonstante kann nicht berechnet werden, weil die benötigten Angaben über den Widerstand $R$ des RC-Gliedes fehlen und keine Messgeräten vorhanden sind.
In der Abbildung \ref{fig:map001} ist gut erkennbar den exponentiellen Abfall bei der Entladevorgang des Kondensators. Mit zunehmender Frequenz nimmt die Amplitude der Kondensatorspannung ab. 
Weiterhin lassen sich auf Messfehler zurückführen. Bei der Einstellung für größere Frequenzen bei der Phasenverschiebung wurden die zu ermittelten Größen immer kleiner und somit war dies ungenau messbar.  
