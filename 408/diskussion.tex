\section{Diskussion}
\label{sec:Diskussion}
Im ersten Versuchsteil sollte die Linsengleichung auf Gültigkeit überprüft werden und mit den Angaben vom Hersteller verglichen werden.

Die Brennweite aus der Abbildung \ref{fig:messungen1} entspricht dabei dem gemessenen Mittelwert $f_1 = \SI{98,5 \pm 0,1 e-3}{\meter}$, der in diesem Fall recht gut mit dem Schnittpunkt der Geraden übereinstimmt. Es führt zu einer geringen Messungenauigkeit. In der Abbildung \ref{fig:messungen2} fällt auf, dass die Geraden sich zum einen weniger deutlich in mehrere Punkten schneiden, als dies in der Abbildung \ref{fig:messungen1} der Fall ist, und diese dann zum anderen deutlich von der berechneten Brennweite $f_2 = \SI{255,4 \pm 16,8 e-3}{\meter}$ abweicht. Dies lässt auf eine große Messungenauigkeit schließen. Der Punkt war ungenau zu bestimmen, da es aufgrund der Zahl an Messwerten mehrere Schnittpunkte gibt und der Punkt bestimmt wurde, um den die meisten der Schnittpunkte lagen.
Die experimentell ermittelte Brennweite $f_1$ weicht um $\SI{1,5}{\percent}$ von der Herstellerangabe ($f_\text{hersteller1} = \SI{100}{\milli\meter}$) ab. Für die zweite Brennweite $f_2$ ergibt sich eine Abweichung vom Herstellerangabe($f_\text{hersteller2} = \SI{150}{\milli\meter}$) von $\SI{70,26}{\percent}$. Die Linsengleichung lässt sich somit nur bei der ersten Wert bestätigen.

Im zweiten Versuchsteil wird die Methode von Bessel verwendet, bei der die Brennweite bestimmen werden sollte und dann diese mit dem Angaben vom Hersteller($f_\text{hersteller1} = \SI{100}{\milli\meter}$) verglichen werden sollte. Analog soll das Verfahren für den roten und blauen Filter verfolgen.
Es ergibt sich eine Abweichung von $\SI{31,9}{\percent}$ vom Theoriewert und das Ergebnis stimmt nicht innerhalb der Fehlergrenzen mit der zuvor berechneten Brennweiten überein, obwohl es dieselbe Linse verwendet wurde.
Für die bestimmten verschiedenen Brennweiten für rotes und blaues Licht kann nicht die chromatische Abberation bestätigt werden, dass blaues Licht aufgrund seiner kürzeren Wellenlänge stärker gebrochen wird als rotes Licht, denn der Wert größer ist als der andere von rotes Licht.

Im letzten Versuchsteil bei der Methode von Abbe war es schwer zu erkennen, wann die Linsen scharf abbildeten. Für diese Methode gab es eine konkave und eine konvexe Linse mit jeweils $\SI{\pm 100 e-3}{\meter}$. Daraus lässt sich erschließen, dass die ausgerechneten Brennweiten ungefähr im erwarteten Bereich der konkaven und konvexen Linse liegen. Die Abweichungen liegen jeweils bei $\SI{23,5}{\percent}$ und $\SI{14,2}{\percent}$.