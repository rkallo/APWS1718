\section{Durchführung}
\label{sec:Durchführung}
\begin{figure}[h!]
	\centering
	\includegraphics[width=0.8\linewidth]{../../aufbau1}
	\caption{Aufbau des verwendeten Lock-In-Verstäkers im ersten Versuchsteil. }
	\label{fig:aufbau1}
\end{figure}

Die verwendete Messaparatur ist modular in der Abbildung \ref{fig:aufbau1} aufgebaut. Damit können die Komponenten einzeln untersucht werden und der Lock-In-Verstärker kann Schritt für Schritt aufgebaut werden. 
Zu bedienen sind: der Vorverstärker, die Filter(Hoch-, Tief- und Bandpaßfilter), der Phasenverschieber, ein Funktionsgenerator, ein Rauschgenerator, ein Tiefpaß-Verstärker und ein Amplituden-/ Lock-In-Detektor. Mit Hilfe des Ozilloskops können dann die Signale einzeln vermessen und schematisch skizziert werden. Der Aufbau wird Schritt für Schritt gemacht, beginnend mit dem Signal Processor und nach jedem neu angeschlossenen Bauteil wird ein Bild des Oszilloskops gemacht. 

Es wird einmal mit verrauschtem und unverrauschtem Signal fünf mal für verschiedene Phasen durchgeführt und das Ausgangssignal nach dem Tiefpaß wird zehn mal in Abhängigkeit von der Phasenverschiebung untersucht. Dabei wurde für diese beiden Teilversuchen eine Frequenz von ca. $\SI{1}{\kilo\hertz}$ und eine Spannung von mindestens $\SI{10}{\milli\volt}$ gestellt.

Im nächsten Versuchteil ist der Aufbau in der Abbildung \ref{fig:ledaufbau} zu sehen. Die Leuchtdiode wird mit einer Rechteckspannung betrieben, die Frequenz liegt in diesem Fall zwischen $\SI{50}{\hertz}$ und $\SI{500}{\hertz}$.
Es wird hier die Lichtintensität in Abhängigkeit des Abstandes $r$ zwischen LED und Photodiode. Hier wird untersucht, wie groß der maximale Abstand $r_\text{max}$ ist, bei dem das Licht der blinkenden Photodiode noch nachgewiesen werden kann. 
\begin{figure}[h!]
	\centering
	\includegraphics[width=0.8\linewidth]{../../LEDAufbau}
	\caption{Aufbau der Photodetektorschaltung. }
	\label{fig:ledaufbau}
\end{figure}
